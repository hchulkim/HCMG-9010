% Options for packages loaded elsewhere
\PassOptionsToPackage{unicode}{hyperref}
\PassOptionsToPackage{hyphens}{url}
\PassOptionsToPackage{dvipsnames,svgnames,x11names}{xcolor}
%
\documentclass[
  letterpaper,
  DIV=11,
  numbers=noendperiod]{scrartcl}

\usepackage{amsmath,amssymb}
\usepackage{iftex}
\ifPDFTeX
  \usepackage[T1]{fontenc}
  \usepackage[utf8]{inputenc}
  \usepackage{textcomp} % provide euro and other symbols
\else % if luatex or xetex
  \usepackage{unicode-math}
  \defaultfontfeatures{Scale=MatchLowercase}
  \defaultfontfeatures[\rmfamily]{Ligatures=TeX,Scale=1}
\fi
\usepackage{lmodern}
\ifPDFTeX\else  
    % xetex/luatex font selection
\fi
% Use upquote if available, for straight quotes in verbatim environments
\IfFileExists{upquote.sty}{\usepackage{upquote}}{}
\IfFileExists{microtype.sty}{% use microtype if available
  \usepackage[]{microtype}
  \UseMicrotypeSet[protrusion]{basicmath} % disable protrusion for tt fonts
}{}
\makeatletter
\@ifundefined{KOMAClassName}{% if non-KOMA class
  \IfFileExists{parskip.sty}{%
    \usepackage{parskip}
  }{% else
    \setlength{\parindent}{0pt}
    \setlength{\parskip}{6pt plus 2pt minus 1pt}}
}{% if KOMA class
  \KOMAoptions{parskip=half}}
\makeatother
\usepackage{xcolor}
\setlength{\emergencystretch}{3em} % prevent overfull lines
\setcounter{secnumdepth}{-\maxdimen} % remove section numbering
% Make \paragraph and \subparagraph free-standing
\makeatletter
\ifx\paragraph\undefined\else
  \let\oldparagraph\paragraph
  \renewcommand{\paragraph}{
    \@ifstar
      \xxxParagraphStar
      \xxxParagraphNoStar
  }
  \newcommand{\xxxParagraphStar}[1]{\oldparagraph*{#1}\mbox{}}
  \newcommand{\xxxParagraphNoStar}[1]{\oldparagraph{#1}\mbox{}}
\fi
\ifx\subparagraph\undefined\else
  \let\oldsubparagraph\subparagraph
  \renewcommand{\subparagraph}{
    \@ifstar
      \xxxSubParagraphStar
      \xxxSubParagraphNoStar
  }
  \newcommand{\xxxSubParagraphStar}[1]{\oldsubparagraph*{#1}\mbox{}}
  \newcommand{\xxxSubParagraphNoStar}[1]{\oldsubparagraph{#1}\mbox{}}
\fi
\makeatother

\usepackage{color}
\usepackage{fancyvrb}
\newcommand{\VerbBar}{|}
\newcommand{\VERB}{\Verb[commandchars=\\\{\}]}
\DefineVerbatimEnvironment{Highlighting}{Verbatim}{commandchars=\\\{\}}
% Add ',fontsize=\small' for more characters per line
\usepackage{framed}
\definecolor{shadecolor}{RGB}{241,243,245}
\newenvironment{Shaded}{\begin{snugshade}}{\end{snugshade}}
\newcommand{\AlertTok}[1]{\textcolor[rgb]{0.68,0.00,0.00}{#1}}
\newcommand{\AnnotationTok}[1]{\textcolor[rgb]{0.37,0.37,0.37}{#1}}
\newcommand{\AttributeTok}[1]{\textcolor[rgb]{0.40,0.45,0.13}{#1}}
\newcommand{\BaseNTok}[1]{\textcolor[rgb]{0.68,0.00,0.00}{#1}}
\newcommand{\BuiltInTok}[1]{\textcolor[rgb]{0.00,0.23,0.31}{#1}}
\newcommand{\CharTok}[1]{\textcolor[rgb]{0.13,0.47,0.30}{#1}}
\newcommand{\CommentTok}[1]{\textcolor[rgb]{0.37,0.37,0.37}{#1}}
\newcommand{\CommentVarTok}[1]{\textcolor[rgb]{0.37,0.37,0.37}{\textit{#1}}}
\newcommand{\ConstantTok}[1]{\textcolor[rgb]{0.56,0.35,0.01}{#1}}
\newcommand{\ControlFlowTok}[1]{\textcolor[rgb]{0.00,0.23,0.31}{\textbf{#1}}}
\newcommand{\DataTypeTok}[1]{\textcolor[rgb]{0.68,0.00,0.00}{#1}}
\newcommand{\DecValTok}[1]{\textcolor[rgb]{0.68,0.00,0.00}{#1}}
\newcommand{\DocumentationTok}[1]{\textcolor[rgb]{0.37,0.37,0.37}{\textit{#1}}}
\newcommand{\ErrorTok}[1]{\textcolor[rgb]{0.68,0.00,0.00}{#1}}
\newcommand{\ExtensionTok}[1]{\textcolor[rgb]{0.00,0.23,0.31}{#1}}
\newcommand{\FloatTok}[1]{\textcolor[rgb]{0.68,0.00,0.00}{#1}}
\newcommand{\FunctionTok}[1]{\textcolor[rgb]{0.28,0.35,0.67}{#1}}
\newcommand{\ImportTok}[1]{\textcolor[rgb]{0.00,0.46,0.62}{#1}}
\newcommand{\InformationTok}[1]{\textcolor[rgb]{0.37,0.37,0.37}{#1}}
\newcommand{\KeywordTok}[1]{\textcolor[rgb]{0.00,0.23,0.31}{\textbf{#1}}}
\newcommand{\NormalTok}[1]{\textcolor[rgb]{0.00,0.23,0.31}{#1}}
\newcommand{\OperatorTok}[1]{\textcolor[rgb]{0.37,0.37,0.37}{#1}}
\newcommand{\OtherTok}[1]{\textcolor[rgb]{0.00,0.23,0.31}{#1}}
\newcommand{\PreprocessorTok}[1]{\textcolor[rgb]{0.68,0.00,0.00}{#1}}
\newcommand{\RegionMarkerTok}[1]{\textcolor[rgb]{0.00,0.23,0.31}{#1}}
\newcommand{\SpecialCharTok}[1]{\textcolor[rgb]{0.37,0.37,0.37}{#1}}
\newcommand{\SpecialStringTok}[1]{\textcolor[rgb]{0.13,0.47,0.30}{#1}}
\newcommand{\StringTok}[1]{\textcolor[rgb]{0.13,0.47,0.30}{#1}}
\newcommand{\VariableTok}[1]{\textcolor[rgb]{0.07,0.07,0.07}{#1}}
\newcommand{\VerbatimStringTok}[1]{\textcolor[rgb]{0.13,0.47,0.30}{#1}}
\newcommand{\WarningTok}[1]{\textcolor[rgb]{0.37,0.37,0.37}{\textit{#1}}}

\providecommand{\tightlist}{%
  \setlength{\itemsep}{0pt}\setlength{\parskip}{0pt}}\usepackage{longtable,booktabs,array}
\usepackage{calc} % for calculating minipage widths
% Correct order of tables after \paragraph or \subparagraph
\usepackage{etoolbox}
\makeatletter
\patchcmd\longtable{\par}{\if@noskipsec\mbox{}\fi\par}{}{}
\makeatother
% Allow footnotes in longtable head/foot
\IfFileExists{footnotehyper.sty}{\usepackage{footnotehyper}}{\usepackage{footnote}}
\makesavenoteenv{longtable}
\usepackage{graphicx}
\makeatletter
\newsavebox\pandoc@box
\newcommand*\pandocbounded[1]{% scales image to fit in text height/width
  \sbox\pandoc@box{#1}%
  \Gscale@div\@tempa{\textheight}{\dimexpr\ht\pandoc@box+\dp\pandoc@box\relax}%
  \Gscale@div\@tempb{\linewidth}{\wd\pandoc@box}%
  \ifdim\@tempb\p@<\@tempa\p@\let\@tempa\@tempb\fi% select the smaller of both
  \ifdim\@tempa\p@<\p@\scalebox{\@tempa}{\usebox\pandoc@box}%
  \else\usebox{\pandoc@box}%
  \fi%
}
% Set default figure placement to htbp
\def\fps@figure{htbp}
\makeatother

\usepackage{float}
\usepackage{tabularray}
\usepackage[normalem]{ulem}
\usepackage{graphicx}
\UseTblrLibrary{booktabs}
\UseTblrLibrary{rotating}
\UseTblrLibrary{siunitx}
\NewTableCommand{\tinytableDefineColor}[3]{\definecolor{#1}{#2}{#3}}
\newcommand{\tinytableTabularrayUnderline}[1]{\underline{#1}}
\newcommand{\tinytableTabularrayStrikeout}[1]{\sout{#1}}
\usepackage{fvextra}
\DefineVerbatimEnvironment{Highlighting}{Verbatim}{breaklines,commandchars=\\\{\}}
\DefineVerbatimEnvironment{OutputCode}{Verbatim}{breaklines,commandchars=\\\{\}}
\KOMAoption{captions}{tableheading}
\makeatletter
\@ifpackageloaded{caption}{}{\usepackage{caption}}
\AtBeginDocument{%
\ifdefined\contentsname
  \renewcommand*\contentsname{Table of contents}
\else
  \newcommand\contentsname{Table of contents}
\fi
\ifdefined\listfigurename
  \renewcommand*\listfigurename{List of Figures}
\else
  \newcommand\listfigurename{List of Figures}
\fi
\ifdefined\listtablename
  \renewcommand*\listtablename{List of Tables}
\else
  \newcommand\listtablename{List of Tables}
\fi
\ifdefined\figurename
  \renewcommand*\figurename{Figure}
\else
  \newcommand\figurename{Figure}
\fi
\ifdefined\tablename
  \renewcommand*\tablename{Table}
\else
  \newcommand\tablename{Table}
\fi
}
\@ifpackageloaded{float}{}{\usepackage{float}}
\floatstyle{ruled}
\@ifundefined{c@chapter}{\newfloat{codelisting}{h}{lop}}{\newfloat{codelisting}{h}{lop}[chapter]}
\floatname{codelisting}{Listing}
\newcommand*\listoflistings{\listof{codelisting}{List of Listings}}
\makeatother
\makeatletter
\makeatother
\makeatletter
\@ifpackageloaded{caption}{}{\usepackage{caption}}
\@ifpackageloaded{subcaption}{}{\usepackage{subcaption}}
\makeatother

\usepackage{bookmark}

\IfFileExists{xurl.sty}{\usepackage{xurl}}{} % add URL line breaks if available
\urlstyle{same} % disable monospaced font for URLs
\hypersetup{
  pdftitle={Problem Set 2},
  pdfauthor={Hyoungchul Kim},
  colorlinks=true,
  linkcolor={blue},
  filecolor={Maroon},
  citecolor={Blue},
  urlcolor={Blue},
  pdfcreator={LaTeX via pandoc}}


\title{Problem Set 2}
\author{Hyoungchul Kim}
\date{2025-03-05}

\begin{document}
\maketitle


First read in some necessary programming packages for analysis:

\begin{Shaded}
\begin{Highlighting}[]
\CommentTok{\# Load libraries}
\ControlFlowTok{if}\NormalTok{ (}\SpecialCharTok{!}\FunctionTok{require}\NormalTok{(pacman)) }\FunctionTok{install.packages}\NormalTok{(}\StringTok{"pacman"}\NormalTok{)}
\FunctionTok{p\_load}\NormalTok{(tidyverse, data.table, fixest, modelsummary, texreg)}
\end{Highlighting}
\end{Shaded}

\section{Theory questions}\label{theory-questions}

\subsection{a.}\label{a.}

We learned in class that 2SLS coefficients in finite sample is biased.
To be exact, Hahn and Hausman (\emph{Ecma}, 2002) showed that
\(\mathbb{E}(b_{2SLS} - \beta) \approx \frac{\sigma_{\eta \xi}}{\sigma^2_\xi} \frac{1}{F+1}\)
where \(F\) is the first stagef-statistics, a proxy for the predictive
strength of the instrument effect. since \(F\) is in the denominator, we
can see that higher predictive power will decrease the bias. On the
other hand, the predictive strength of the instrument does not affect
the consistency of the IV estimator. As long as we have the exclusion
condition satisfied, the probability limit of the IV estimator converges
to the estimand.

\subsection{b.}\label{b.}

Since LATE is still a type of IV, we need the exclusion and relativity
condition to hold. That is, the IV should be correlated with the
treatment varible of interest and should only affect the outcome through
the treatment variable. (Note that SUTVA, random assignment is also
necessary for the IV method to hold).

One extra assumption we need is the \textbf{monotonicity condition}.
This just means that there are no defier.

Now let's prove the LATE theorem under these maintained assumptions:

Using assumptions above, we can get:

\begin{align}
  \beta_{late} &= \frac{\mathbb{E}(Y_i \mid Z_i = 1) - \mathbb{E}(Y_i \mid Z_i = 0)}{\mathbb{E}(D_i \mid Z_i = 1) - \mathbb{E}(D_i \mid Z_i = 0)}\\
   &= \frac{\mathbb{E}[Y_{1i} - Y_{0i} \mid \text{compliers}]P(\text{compliers})}{P(\text{compliers})}\\
   &= \mathbb{E}[Y_{1i} - Y_{0i} \mid \text{compliers}]
\end{align}

So this shows that by using the assumption we can get the ATE for the
compliers in our data.

\subsection{c.}\label{c.}

You can easily check from the question b that the monotonicity is
crucial for retrieving LATE. The nominator in the question b is the
reduced form estimate. Without monotonicity, we cannot have the reduced
form estimate to equal the change in potential outcome effect for
compliers. You can also see that the denominator implies the usefulness
of monotonicity in first stage. If we assume there is defiers, the first
stage estimate cannot be the probability of being the complier. Thus
monotonicity is crucial to pin down the LATE estimate.

\subsection{d.}\label{d.}

If there is no always takers, this means all of our data will be made up
of compliers (under the assumption of monotonicity). So this implies
that \(\mathbb{E} [D_i \mid Z_i = 0] = 0\). Then using the fact that
\(\mathbb{E}[Y_i \mid Z_i = 1] = \mathbb{E}[Y_{0i}]\) and
\(\mathbb{E}[Y_i \mid Z_i = 1] = \mathbb{E}[Y_{0i} + (Y_{1i} - Y_{0i}) D_i \mid Z_i = 1]\)
from the lecture, we get:

\begin{align}
 \beta_{late} &= \frac{\mathbb{E} [(Y_{1i}-Y_{0i}) D_i \mid Z_i = 1]}{P[D_i = 1 \mid Z_i =1]}\\
  &= \frac{\mathbb{E}[(Y_{1i} - Y_{0i}) (1) \mid D_i =1, Z_i =1] P[D_i = 1 \mid Z_i = 1]+ 0}{P[D_i = 1 \mid Z_i = 1]}\\
  &= \mathbb{E}[(Y_{1i} - Y_{0i} \mid D_i =1, Z_i = 1)]\\
  &= \mathbb{E}[(Y_{1i} - Y_{0i}) \mid D_i = 1] = ATT
\end{align}

\subsection{e.}\label{e.}

Unlike LATE, MTC allows sorting on unobserved treatment effects. For
LATE, if there is unobserved soring, it might not recover the ATE for
compliers. MTA in some way alleviates this issue by allowing treatment
probability to be correlated with treatment effects. It is also helpful
if you want to recover population treatment effect parameters like ATT
and not jus the LATE. But we need to assume some functional form
restrictions. For example, we assume treatment effect is additive and
separable. Also, we assume outcome Y depends on X in a linear, additive
form. Using these assumptions, we can write MTC as sum of observed and
unobserved components:
\(MTE(X=x, U_D = p) = x'(\beta_1 - \beta_0) + \mathbb{E}(u_{1i} - u_{0i} \mid p)\).
Then we can express
\(\mathbb{E}[Y_i \mid X = x, P = p] = x' \beta_0 + x' (\beta_1 - \beta_0)p + K(p)\),
where laster term on RHS is a polynomial that approximates the
unobserved portion of Y. Then we can estimate the MTE by taking the
derivative with respect to p.

\section{Numerical questions}\label{numerical-questions}

\subsection{a.}\label{a.-1}

Now we get the balance table by treatment status.

\begin{table}[!h]
\centering
\resizebox{\textwidth}{!}{
\begin{tabular}[t]{lrrrrrr}
\toprule
Variable & Control\_Mean & Treatment\_Mean & Difference & Control\_SD & Treatment\_SD & SE\\
\midrule
Birth Year & 1968.0367979 & 1968.2898936 & 0.2530956 & 0.0934849 & 0.0926367 & 0.1316151\\
Female & 0.5563614 & 0.5391469 & -0.0172144 & 0.0038371 & 0.0038168 & 0.0054126\\
English Materials Requested & 0.9085177 & 0.9020445 & -0.0064732 & 0.0022266 & 0.0022760 & 0.0031853\\
Signed Self Up & 0.8763579 & 0.8324037 & -0.0439542 & 0.0025423 & 0.0028599 & 0.0038320\\
Signed Up First Day & 0.0900322 & 0.0953294 & 0.0052972 & 0.0022107 & 0.0022486 & 0.0031543\\
\addlinespace
Gave Phone Number & 0.8706070 & 0.8686516 & -0.0019555 & 0.0025923 & 0.0025863 & 0.0036623\\
Gave PO Box as Address & 0.1145806 & 0.1168259 & 0.0022453 & 0.0024600 & 0.0024595 & 0.0034791\\
ZIP is an MSA & 0.7693568 & 0.7619855 & -0.0073713 & 0.0032535 & 0.0032608 & 0.0046071\\
\bottomrule
\end{tabular}}
\end{table}

We can see that there is not much significant differences between
control group and treatment group. This seems to imply that
randomization is quite well done.

Now let's check ITT:

\begin{table}[!h]
\resizebox{\textwidth}{!}{
\centering
\begin{tabular}[t]{lrr}
\toprule
Variable & ITT\_Estimate & SE\\
\midrule
Any Prescription in 12 Months & 0.0121 & 0.0071\\
Any Doctor Visit in 12 Months & 0.0555 & 0.0064\\
Any ER Visit in 12 Months & -0.0007 & 0.0057\\
Any Hospitalization in 12 Months & 0.0000 & 0.0034\\
\bottomrule
\end{tabular}}
\end{table}

Somehow I am getting bit weird coefficient. This should be all positive
as winning the lottery will make people more likely to get a health care
insurance.

\clearpage

\subsection{b.}\label{b.-1}

Using OLS is problematic as people receive medicaid and people who don't
are different set of people. That is, the potential outcome (health)
will be correlated with the treatment of having medicaid. Thus running
OLS is give us a biased result.

Let's look at a balance table between those who received medicaid and
the remaining people.

\clearpage

\begin{table}
\centering
\begin{talltblr}[         %% tabularray outer open
caption={Balance table (survey responders only)},
]                     %% tabularray outer close
{                     %% tabularray inner open
colspec={Q[]Q[]Q[]},
column{1}={}{halign=l,},
column{2,3}={}{halign=r,},
}                     %% tabularray inner close
\toprule
& No medicaid & Yes medicaid \\ \midrule %% TinyTableHeader
birthyear\_list            & \num{1968.322} & \num{1965.767} \\
Female                      & \num{0.539}    & \num{0.630}    \\
English materials requested & \num{0.905}    & \num{0.931}    \\
ZIP is an MSA               & \num{0.767}    & \num{0.753}    \\
Average hrs worked/week     & \num{2.208}    & \num{1.624}    \\
\bottomrule
\end{talltblr}
\end{table}

You can see some characteristics differ between two groups than the
previous randomization (gender ratio, average hours worked, etc).

\subsection{c.}\label{c.-1}

Let's run the result. For indicator of medicaid, we will use currently
on medicaid, ever on medicaid as the variable. The coefficients in this
case would be ATE but as there is selection problem, they would be
biased. First result used Every on mediciad as the variable.

\begin{verbatim}

===========================================================================
                       Current prescription  Any primary care  Any ER      
---------------------------------------------------------------------------
Ever on mediciad           0.12 ***              0.16 ***          0.05 ***
                          (0.01)                (0.01)            (0.01)   
---------------------------------------------------------------------------
Num. obs.              18308                 23492             23514       
R^2 (full model)           0.01                  0.02              0.00    
R^2 (proj model)                                                           
Adj. R^2 (full model)      0.01                  0.02              0.00    
Adj. R^2 (proj model)                                                      
===========================================================================
*** p < 0.001; ** p < 0.01; * p < 0.05
\end{verbatim}

Now I will use currently on medicaid as the variable.

\begin{verbatim}

===========================================================================
                       Current prescription  Any primary care  Any ER      
---------------------------------------------------------------------------
Currently on medicaid      0.20 ***              0.32 ***          0.06 ***
                          (0.01)                (0.01)            (0.01)   
---------------------------------------------------------------------------
Num. obs.              18047                 23157             23177       
R^2 (full model)           0.04                  0.10              0.00    
R^2 (proj model)                                                           
Adj. R^2 (full model)      0.04                  0.10              0.00    
Adj. R^2 (proj model)                                                      
===========================================================================
*** p < 0.001; ** p < 0.01; * p < 0.05
\end{verbatim}

\subsection{d.}\label{d.-1}

For monotonicity, I guess it would be assuming that people who got into
lottery will not have tendency to not get a health insurance. I think
monotonicity would hold in this case as the lottery is just giving you
chance to apply for health care (OHS Standard). Thus it will at least
weakly make people apply for health care and get one.

Also, exclusion in this case would mean the lottery will only affect the
outcome (health and hospital related outcome) only through the treatment
(getting a health care). This also seems reasonable as lottery is just
like an experiment. Thus it is very unlikely it will directly affect the
outcome.

\subsection{e.}\label{e.-1}

We run the fist stage regrssion result using two possible measures of
medicaid coverage.

\begin{table}[!h]
\resizebox{\textwidth}{!}{
\centering
\begin{tabular}[t]{lrrr}
\toprule
Variable & Control\_Mean & First\_Stage & SE\\
\midrule
Ever on Medicaid & 0.141 & 0.2565 & 0.0046\\
Currently on Medicaid & 0.053 & 0.0907 & 0.0032\\
\bottomrule
\end{tabular}}
\end{table}

\subsection{f.}\label{f.}

Compliers in our case are people who actually receive medicaid after
they win the lottery, but would not have gotten it if they had not won
the lottery. To characterize the complier group, we can try to
difference the means of medicaid variables for lottery winners and
losers. This is somewhat like a first stage estimates. The table below
shows the complier share for two medicaid measures.

We can also use background info on people to characterize the
composition of the complier group. This can be thought of as first stage
coefficient of people with \(X_i=1\) (people with certain
characteristics) divided by the coefficient of the full sample.I wrote
down some values for few demographics. It seems that complier group is
more older less likely to be female, and signed up on the first day more
often.

\begin{table}[!h]
\centering
\begin{tabular}[t]{lrr}
\toprule
Demographic & Ever on Medicaid & Currently on Medicaid\\
\midrule
Share of Compliers & 0.257 & 0.189\\
old (Born before 1968) & 1.144 & 1.160\\
female & 0.975 & 0.947\\
signed self up & 1.067 & 1.089\\
\bottomrule
\end{tabular}
\end{table}

\subsection{g.}\label{g.}

We will get the LATE estimates using 2SLS regression. This is ATE result
for the compliers only. If we assume there is heterogeneous effects, we
should not think of this estimate as the ATT.

\begin{table}[!h]
\resizebox{\textwidth}{!}{
\centering
\begin{tabular}[t]{lrr}
\toprule
Variable & LATE & SE\\
\midrule
Any Prescription in 12 Months & 0.0422 & 0.0246\\
Any Doctor Visit in 12 Months & 0.1908 & 0.0216\\
Any ER Visit in 12 Months & -0.0023 & 0.0197\\
Any Hospitalization in 12 Months & 0.0001 & 0.0116\\
\bottomrule
\end{tabular}}
\end{table}

\section{R codes}\label{r-codes}

\begin{Shaded}
\begin{Highlighting}[]
\CommentTok{\# downloaded data from "https://data.nber.org/oregon/4.data.html" and unzipped it.}

\CommentTok{\# Using "oregon\_hie\_qje\_replication.do" file, I copy data files from input folder to the data folder under name "repl\_code".}



\DocumentationTok{\#\#\#\#\#\#\#\#\#\#\#\#\#\#\#\#\#\#\#\#\#\#\#\#\#\#\#\#\#\#\#\#\#\#\#\#\#\#\#\#\#\#\#\#\#\#\#\#\#\#\#\#\#\#\#\#\#\#\#\#\#\#\#\#\#\#\#\#\#\#\#\#\#\#\#\#\#\#\#\#}
\CommentTok{\# AUTHORS: Hyoungchul Kim}
\CommentTok{\# CREATED: 2023{-}02{-}24}
\CommentTok{\# PURPOSE: Solve coding portions of HCMG 901 Problem Set 2 using tidyverse}
\DocumentationTok{\#\#\#\#\#\#\#\#\#\#\#\#\#\#\#\#\#\#\#\#\#\#\#\#\#\#\#\#\#\#\#\#\#\#\#\#\#\#\#\#\#\#\#\#\#\#\#\#\#\#\#\#\#\#\#\#\#\#\#\#\#\#\#\#\#\#\#\#\#\#\#\#\#\#\#\#\#\#\#\#}

\FunctionTok{library}\NormalTok{(tidyverse)}
\FunctionTok{library}\NormalTok{(haven)}
\FunctionTok{library}\NormalTok{(fixest)}
\FunctionTok{library}\NormalTok{(modelsummary)}
\FunctionTok{library}\NormalTok{(kableExtra)}

\CommentTok{\# Load Data}
\NormalTok{analysis\_data }\OtherTok{\textless{}{-}} \FunctionTok{read\_dta}\NormalTok{(}\StringTok{"repl\_data/data\_for\_analysis.dta"}\NormalTok{)}


\CommentTok{\# Define variable lists with labels}
\NormalTok{variable\_labels }\OtherTok{\textless{}{-}} \FunctionTok{c}\NormalTok{(}
  \AttributeTok{birthyear\_list =} \StringTok{"Birth Year"}\NormalTok{,}
  \AttributeTok{female\_list =} \StringTok{"Female"}\NormalTok{,}
  \AttributeTok{english\_list =} \StringTok{"English Materials Requested"}\NormalTok{,}
  \AttributeTok{self\_list =} \StringTok{"Signed Self Up"}\NormalTok{,}
  \AttributeTok{first\_day\_list =} \StringTok{"Signed Up First Day"}\NormalTok{,}
  \AttributeTok{have\_phone\_list =} \StringTok{"Gave Phone Number"}\NormalTok{,}
  \AttributeTok{pobox\_list =} \StringTok{"Gave PO Box as Address"}\NormalTok{,}
  \AttributeTok{zip\_msa =} \StringTok{"ZIP is an MSA"}\NormalTok{,}
  \AttributeTok{rx\_any\_12m =} \StringTok{"Any Prescription in 12 Months"}\NormalTok{,}
  \AttributeTok{doc\_any\_12m =} \StringTok{"Any Doctor Visit in 12 Months"}\NormalTok{,}
  \AttributeTok{er\_any\_12m =} \StringTok{"Any ER Visit in 12 Months"}\NormalTok{,}
  \AttributeTok{hosp\_any\_12m =} \StringTok{"Any Hospitalization in 12 Months"}\NormalTok{,}
  \AttributeTok{ohp\_all\_ever\_survey =} \StringTok{"Ever on Medicaid"}\NormalTok{,}
  \AttributeTok{ohp\_all\_at\_12m =} \StringTok{"Currently on Medicaid"}
\NormalTok{)}

\NormalTok{baseline\_list }\OtherTok{\textless{}{-}} \FunctionTok{names}\NormalTok{(variable\_labels)[}\DecValTok{1}\SpecialCharTok{:}\DecValTok{8}\NormalTok{]}
\NormalTok{survey\_useext\_list }\OtherTok{\textless{}{-}} \FunctionTok{names}\NormalTok{(variable\_labels)[}\DecValTok{9}\SpecialCharTok{:}\DecValTok{12}\NormalTok{]}
\NormalTok{mdcd\_covg\_vars }\OtherTok{\textless{}{-}} \FunctionTok{names}\NormalTok{(variable\_labels)[}\DecValTok{13}\SpecialCharTok{:}\DecValTok{14}\NormalTok{]}

\CommentTok{\# Function to save results as LaTeX tables with labels}
\NormalTok{save\_tex }\OtherTok{\textless{}{-}} \ControlFlowTok{function}\NormalTok{(data, filename) \{}
\NormalTok{  data }\OtherTok{\textless{}{-}}\NormalTok{ data }\SpecialCharTok{\%\textgreater{}\%} 
    \FunctionTok{mutate}\NormalTok{(}\AttributeTok{Variable =} \FunctionTok{recode}\NormalTok{(Variable, }\SpecialCharTok{!!!}\NormalTok{variable\_labels)) }\SpecialCharTok{\%\textgreater{}\%} 
    \FunctionTok{mutate}\NormalTok{(}\FunctionTok{across}\NormalTok{(}\FunctionTok{where}\NormalTok{(is.numeric), }\SpecialCharTok{\textasciitilde{}} \FunctionTok{round}\NormalTok{(.x, }\DecValTok{4}\NormalTok{)))  }\CommentTok{\# Round all numeric values to 2 decimal places}
\NormalTok{  tex\_output }\OtherTok{\textless{}{-}}\NormalTok{ data }\SpecialCharTok{\%\textgreater{}\%}
    \FunctionTok{kbl}\NormalTok{(}\AttributeTok{format =} \StringTok{"latex"}\NormalTok{, }\AttributeTok{booktabs =} \ConstantTok{TRUE}\NormalTok{, }\AttributeTok{caption =}\NormalTok{ filename) }\SpecialCharTok{\%\textgreater{}\%}
    \FunctionTok{kable\_styling}\NormalTok{(}\AttributeTok{latex\_options =} \FunctionTok{c}\NormalTok{(}\StringTok{"hold\_position"}\NormalTok{))}
  
  \FunctionTok{writeLines}\NormalTok{(tex\_output, filename)}
\NormalTok{\}}



\CommentTok{\# Balance Table by Lottery Outcome}
\NormalTok{balance\_results }\OtherTok{\textless{}{-}} \FunctionTok{map\_dfr}\NormalTok{(baseline\_list, }\ControlFlowTok{function}\NormalTok{(var) \{}
\NormalTok{  filtered\_data }\OtherTok{\textless{}{-}}\NormalTok{ analysis\_data }\SpecialCharTok{\%\textgreater{}\%}
    \FunctionTok{filter}\NormalTok{(}\SpecialCharTok{!}\FunctionTok{is.na}\NormalTok{(weight\_12m), }\SpecialCharTok{!}\FunctionTok{is.na}\NormalTok{(.data[[var]]))}
  
  \ControlFlowTok{if}\NormalTok{ (}\FunctionTok{nrow}\NormalTok{(filtered\_data) }\SpecialCharTok{==} \DecValTok{0}\NormalTok{) }\FunctionTok{return}\NormalTok{(}\ConstantTok{NULL}\NormalTok{)}
  
\NormalTok{  control\_stats }\OtherTok{\textless{}{-}} \FunctionTok{feols}\NormalTok{(}\FunctionTok{as.formula}\NormalTok{(}\FunctionTok{paste}\NormalTok{(var, }\StringTok{"\textasciitilde{} 1"}\NormalTok{)), }
                         \AttributeTok{data =} \FunctionTok{filter}\NormalTok{(filtered\_data, treatment }\SpecialCharTok{==} \DecValTok{0}\NormalTok{), }
                         \AttributeTok{weights =} \SpecialCharTok{\textasciitilde{}}\NormalTok{ weight\_12m)}
  
\NormalTok{  treatment\_stats }\OtherTok{\textless{}{-}} \FunctionTok{feols}\NormalTok{(}\FunctionTok{as.formula}\NormalTok{(}\FunctionTok{paste}\NormalTok{(var, }\StringTok{"\textasciitilde{} 1"}\NormalTok{)), }
                           \AttributeTok{data =} \FunctionTok{filter}\NormalTok{(filtered\_data, treatment }\SpecialCharTok{==} \DecValTok{1}\NormalTok{), }
                           \AttributeTok{weights =} \SpecialCharTok{\textasciitilde{}}\NormalTok{ weight\_12m)}
  
\NormalTok{  diff\_stats }\OtherTok{\textless{}{-}} \FunctionTok{feols}\NormalTok{(}\FunctionTok{as.formula}\NormalTok{(}\FunctionTok{paste}\NormalTok{(var, }\StringTok{"\textasciitilde{} treatment"}\NormalTok{)), }
                      \AttributeTok{data =}\NormalTok{ filtered\_data, }
                      \AttributeTok{weights =} \SpecialCharTok{\textasciitilde{}}\NormalTok{ weight\_12m)}
  
  \FunctionTok{tibble}\NormalTok{(}
    \AttributeTok{Variable =}\NormalTok{ var,}
    \AttributeTok{Control\_Mean =} \FunctionTok{coef}\NormalTok{(control\_stats)[}\DecValTok{1}\NormalTok{],}
    \AttributeTok{Treatment\_Mean =} \FunctionTok{coef}\NormalTok{(treatment\_stats)[}\DecValTok{1}\NormalTok{],}
    \AttributeTok{Difference =} \FunctionTok{coef}\NormalTok{(diff\_stats)[}\DecValTok{2}\NormalTok{],}
    \AttributeTok{Control\_SD =} \FunctionTok{sqrt}\NormalTok{(}\FunctionTok{vcov}\NormalTok{(control\_stats)[}\DecValTok{1}\NormalTok{,}\DecValTok{1}\NormalTok{]),}
    \AttributeTok{Treatment\_SD =} \FunctionTok{sqrt}\NormalTok{(}\FunctionTok{vcov}\NormalTok{(treatment\_stats)[}\DecValTok{1}\NormalTok{,}\DecValTok{1}\NormalTok{]),}
    \AttributeTok{SE =} \FunctionTok{sqrt}\NormalTok{(}\FunctionTok{vcov}\NormalTok{(diff\_stats)[}\DecValTok{2}\NormalTok{,}\DecValTok{2}\NormalTok{])}
\NormalTok{  )}
\NormalTok{\})}
\FunctionTok{save\_tex}\NormalTok{(balance\_results, }\StringTok{"tab\_a\_balance.tex"}\NormalTok{)}

\CommentTok{\# ITT Effects of Lottery on Healthcare Usage}
\NormalTok{itt\_results }\OtherTok{\textless{}{-}} \FunctionTok{map\_dfr}\NormalTok{(survey\_useext\_list, }\ControlFlowTok{function}\NormalTok{(var) \{}
\NormalTok{  filtered\_data }\OtherTok{\textless{}{-}}\NormalTok{ analysis\_data }\SpecialCharTok{\%\textgreater{}\%} \FunctionTok{filter}\NormalTok{(}\SpecialCharTok{!}\FunctionTok{is.na}\NormalTok{(weight\_12m), }\SpecialCharTok{!}\FunctionTok{is.na}\NormalTok{(.data[[var]]))}
  \ControlFlowTok{if}\NormalTok{ (}\FunctionTok{nrow}\NormalTok{(filtered\_data) }\SpecialCharTok{==} \DecValTok{0}\NormalTok{) }\FunctionTok{return}\NormalTok{(}\ConstantTok{NULL}\NormalTok{)}
  
\NormalTok{  model }\OtherTok{\textless{}{-}} \FunctionTok{feols}\NormalTok{(}\FunctionTok{as.formula}\NormalTok{(}\FunctionTok{paste}\NormalTok{(var, }\StringTok{"\textasciitilde{} treatment"}\NormalTok{)), }
                 \AttributeTok{data =}\NormalTok{ filtered\_data, }
                 \AttributeTok{weights =} \SpecialCharTok{\textasciitilde{}}\NormalTok{ weight\_12m)}
  
  \FunctionTok{tibble}\NormalTok{(}
    \AttributeTok{Variable =}\NormalTok{ var,}
    \AttributeTok{ITT\_Estimate =} \FunctionTok{coef}\NormalTok{(model)[}\DecValTok{2}\NormalTok{],}
    \AttributeTok{SE =} \FunctionTok{sqrt}\NormalTok{(}\FunctionTok{vcov}\NormalTok{(model)[}\DecValTok{2}\NormalTok{,}\DecValTok{2}\NormalTok{])}
\NormalTok{  )}
\NormalTok{\})}
\FunctionTok{save\_tex}\NormalTok{(itt\_results, }\StringTok{"tab\_a\_itt.tex"}\NormalTok{)}

\CommentTok{\# First{-}Stage Estimates}
\NormalTok{first\_stage\_results }\OtherTok{\textless{}{-}} \FunctionTok{map\_dfr}\NormalTok{(mdcd\_covg\_vars, }\ControlFlowTok{function}\NormalTok{(var) \{}
\NormalTok{  filtered\_data }\OtherTok{\textless{}{-}}\NormalTok{ analysis\_data }\SpecialCharTok{\%\textgreater{}\%} \FunctionTok{filter}\NormalTok{(}\SpecialCharTok{!}\FunctionTok{is.na}\NormalTok{(weight\_12m), }\SpecialCharTok{!}\FunctionTok{is.na}\NormalTok{(.data[[var]]))}
  \ControlFlowTok{if}\NormalTok{ (}\FunctionTok{nrow}\NormalTok{(filtered\_data) }\SpecialCharTok{==} \DecValTok{0}\NormalTok{) }\FunctionTok{return}\NormalTok{(}\ConstantTok{NULL}\NormalTok{)}
  
\NormalTok{  model }\OtherTok{\textless{}{-}} \FunctionTok{feols}\NormalTok{(}\FunctionTok{as.formula}\NormalTok{(}\FunctionTok{paste}\NormalTok{(var, }\StringTok{"\textasciitilde{} treatment"}\NormalTok{)), }
                 \AttributeTok{data =}\NormalTok{ filtered\_data, }
                 \AttributeTok{weights =} \SpecialCharTok{\textasciitilde{}}\NormalTok{ weight\_12m)}
  
  \FunctionTok{tibble}\NormalTok{(}
    \AttributeTok{Variable =}\NormalTok{ var,}
    \AttributeTok{Control\_Mean =} \FunctionTok{coef}\NormalTok{(model)[}\DecValTok{1}\NormalTok{],}
    \AttributeTok{First\_Stage =} \FunctionTok{coef}\NormalTok{(model)[}\DecValTok{2}\NormalTok{],}
    \AttributeTok{SE =} \FunctionTok{sqrt}\NormalTok{(}\FunctionTok{vcov}\NormalTok{(model)[}\DecValTok{2}\NormalTok{,}\DecValTok{2}\NormalTok{])}
\NormalTok{  )}
\NormalTok{\})}
\FunctionTok{save\_tex}\NormalTok{(first\_stage\_results, }\StringTok{"tab\_e.tex"}\NormalTok{)}

\CommentTok{\# 2SLS Estimation for LATE}
\NormalTok{late\_results }\OtherTok{\textless{}{-}} \FunctionTok{map\_dfr}\NormalTok{(survey\_useext\_list, }\ControlFlowTok{function}\NormalTok{(var) \{}
\NormalTok{  filtered\_data }\OtherTok{\textless{}{-}}\NormalTok{ analysis\_data }\SpecialCharTok{\%\textgreater{}\%} \FunctionTok{filter}\NormalTok{(}\SpecialCharTok{!}\FunctionTok{is.na}\NormalTok{(weight\_12m), }\SpecialCharTok{!}\FunctionTok{is.na}\NormalTok{(.data[[var]]))}
  \ControlFlowTok{if}\NormalTok{ (}\FunctionTok{nrow}\NormalTok{(filtered\_data) }\SpecialCharTok{==} \DecValTok{0}\NormalTok{) }\FunctionTok{return}\NormalTok{(}\ConstantTok{NULL}\NormalTok{)}
  
\NormalTok{  model }\OtherTok{\textless{}{-}} \FunctionTok{feols}\NormalTok{(}\FunctionTok{as.formula}\NormalTok{(}\FunctionTok{paste}\NormalTok{(var, }\StringTok{"\textasciitilde{} 1 | ohp\_all\_ever\_survey \textasciitilde{} treatment"}\NormalTok{)), }
                 \AttributeTok{data =}\NormalTok{ filtered\_data, }
                 \AttributeTok{weights =} \SpecialCharTok{\textasciitilde{}}\NormalTok{ weight\_12m)}
  
  \FunctionTok{tibble}\NormalTok{(}
    \AttributeTok{Variable =}\NormalTok{ var,}
    \AttributeTok{LATE =} \FunctionTok{coef}\NormalTok{(model)[}\DecValTok{2}\NormalTok{],}
    \AttributeTok{SE =} \FunctionTok{sqrt}\NormalTok{(}\FunctionTok{vcov}\NormalTok{(model)[}\DecValTok{2}\NormalTok{,}\DecValTok{2}\NormalTok{])}
\NormalTok{  )}
\NormalTok{\})}
\FunctionTok{save\_tex}\NormalTok{(late\_results, }\StringTok{"tab\_g.tex"}\NormalTok{)}




\CommentTok{\# Compliers Analysis}
\NormalTok{analysis\_data }\OtherTok{\textless{}{-}}\NormalTok{ analysis\_data }\SpecialCharTok{\%\textgreater{}\%}
  \FunctionTok{mutate}\NormalTok{(}
    \AttributeTok{old =}\NormalTok{ birthyear\_list }\SpecialCharTok{\textless{}} \DecValTok{1968}\NormalTok{,}
    \AttributeTok{fem =}\NormalTok{ female\_list,}
    \AttributeTok{self =}\NormalTok{ self\_list}
\NormalTok{  )}

\NormalTok{dem\_vars }\OtherTok{\textless{}{-}} \FunctionTok{c}\NormalTok{(}\StringTok{"old"}\NormalTok{, }\StringTok{"fem"}\NormalTok{, }\StringTok{"self"}\NormalTok{)}
\NormalTok{RHS\_vars }\OtherTok{\textless{}{-}} \FunctionTok{c}\NormalTok{(}\StringTok{"ohp\_std\_ever\_survey"}\NormalTok{, }\StringTok{"ins\_any\_12m"}\NormalTok{)}

\NormalTok{complier\_results }\OtherTok{\textless{}{-}} \FunctionTok{map\_dfr}\NormalTok{(RHS\_vars, }\ControlFlowTok{function}\NormalTok{(RHS\_var) \{}
\NormalTok{  treated }\OtherTok{\textless{}{-}}\NormalTok{ analysis\_data }\SpecialCharTok{\%\textgreater{}\%} \FunctionTok{filter}\NormalTok{(treatment }\SpecialCharTok{==} \DecValTok{1}\NormalTok{) }\SpecialCharTok{\%\textgreater{}\%} \FunctionTok{summarise}\NormalTok{(}\AttributeTok{mean =} \FunctionTok{mean}\NormalTok{(.data[[RHS\_var]], }\AttributeTok{na.rm =} \ConstantTok{TRUE}\NormalTok{)) }\SpecialCharTok{\%\textgreater{}\%} \FunctionTok{pull}\NormalTok{(mean)}
\NormalTok{  untreated }\OtherTok{\textless{}{-}}\NormalTok{ analysis\_data }\SpecialCharTok{\%\textgreater{}\%} \FunctionTok{filter}\NormalTok{(treatment }\SpecialCharTok{==} \DecValTok{0}\NormalTok{) }\SpecialCharTok{\%\textgreater{}\%} \FunctionTok{summarise}\NormalTok{(}\AttributeTok{mean =} \FunctionTok{mean}\NormalTok{(.data[[RHS\_var]], }\AttributeTok{na.rm =} \ConstantTok{TRUE}\NormalTok{)) }\SpecialCharTok{\%\textgreater{}\%} \FunctionTok{pull}\NormalTok{(mean)}
\NormalTok{  csize }\OtherTok{\textless{}{-}} \FunctionTok{round}\NormalTok{(treated }\SpecialCharTok{{-}}\NormalTok{ untreated, }\DecValTok{3}\NormalTok{)}
  
\NormalTok{  frac\_results }\OtherTok{\textless{}{-}} \FunctionTok{map\_dfr}\NormalTok{(dem\_vars, }\ControlFlowTok{function}\NormalTok{(var) \{}
\NormalTok{    treated\_var }\OtherTok{\textless{}{-}}\NormalTok{ analysis\_data }\SpecialCharTok{\%\textgreater{}\%} \FunctionTok{filter}\NormalTok{(treatment }\SpecialCharTok{==} \DecValTok{1}\NormalTok{, .data[[var]] }\SpecialCharTok{==} \DecValTok{1}\NormalTok{) }\SpecialCharTok{\%\textgreater{}\%} \FunctionTok{summarise}\NormalTok{(}\AttributeTok{mean =} \FunctionTok{mean}\NormalTok{(.data[[RHS\_var]], }\AttributeTok{na.rm =} \ConstantTok{TRUE}\NormalTok{)) }\SpecialCharTok{\%\textgreater{}\%} \FunctionTok{pull}\NormalTok{(mean)}
\NormalTok{    untreated\_var }\OtherTok{\textless{}{-}}\NormalTok{ analysis\_data }\SpecialCharTok{\%\textgreater{}\%} \FunctionTok{filter}\NormalTok{(treatment }\SpecialCharTok{==} \DecValTok{0}\NormalTok{, .data[[var]] }\SpecialCharTok{==} \DecValTok{1}\NormalTok{) }\SpecialCharTok{\%\textgreater{}\%} \FunctionTok{summarise}\NormalTok{(}\AttributeTok{mean =} \FunctionTok{mean}\NormalTok{(.data[[RHS\_var]], }\AttributeTok{na.rm =} \ConstantTok{TRUE}\NormalTok{)) }\SpecialCharTok{\%\textgreater{}\%} \FunctionTok{pull}\NormalTok{(mean)}
\NormalTok{    frac }\OtherTok{\textless{}{-}} \FunctionTok{round}\NormalTok{((treated\_var }\SpecialCharTok{{-}}\NormalTok{ untreated\_var) }\SpecialCharTok{/}\NormalTok{ csize, }\DecValTok{3}\NormalTok{)}
    \FunctionTok{tibble}\NormalTok{(}\AttributeTok{Demographic =}\NormalTok{ var, }\AttributeTok{Outcome =}\NormalTok{ RHS\_var, }\AttributeTok{Fraction =}\NormalTok{ frac)}
\NormalTok{  \})}
  \FunctionTok{bind\_rows}\NormalTok{(}\FunctionTok{tibble}\NormalTok{(}\AttributeTok{Demographic =} \StringTok{"Share Compliers"}\NormalTok{, }\AttributeTok{Outcome =}\NormalTok{ RHS\_var, }\AttributeTok{Fraction =}\NormalTok{ csize), frac\_results)}
\NormalTok{\})}

\CommentTok{\# Save as LaTeX}
\NormalTok{complier\_tex }\OtherTok{\textless{}{-}}\NormalTok{ complier\_results }\SpecialCharTok{\%\textgreater{}\%}
  \FunctionTok{pivot\_wider}\NormalTok{(}\AttributeTok{names\_from =}\NormalTok{ Outcome, }\AttributeTok{values\_from =}\NormalTok{ Fraction) }\SpecialCharTok{\%\textgreater{}\%}
  \FunctionTok{mutate}\NormalTok{(}\AttributeTok{Demographic =} \FunctionTok{recode}\NormalTok{(Demographic, }\SpecialCharTok{!!!}\NormalTok{variable\_labels)) }\SpecialCharTok{\%\textgreater{}\%}
  \FunctionTok{kbl}\NormalTok{(}\AttributeTok{format =} \StringTok{"latex"}\NormalTok{, }\AttributeTok{booktabs =} \ConstantTok{TRUE}\NormalTok{, }\AttributeTok{caption =} \StringTok{"Meet The Compliers"}\NormalTok{) }\SpecialCharTok{\%\textgreater{}\%}
  \FunctionTok{kable\_styling}\NormalTok{(}\AttributeTok{latex\_options =} \StringTok{"hold\_position"}\NormalTok{)}

\FunctionTok{writeLines}\NormalTok{(complier\_tex, }\StringTok{"QDc\_compliers.tex"}\NormalTok{) }
\end{Highlighting}
\end{Shaded}





\end{document}
