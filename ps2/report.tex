% Options for packages loaded elsewhere
\PassOptionsToPackage{unicode}{hyperref}
\PassOptionsToPackage{hyphens}{url}
\PassOptionsToPackage{dvipsnames,svgnames,x11names}{xcolor}
%
\documentclass[
  letterpaper,
  DIV=11,
  numbers=noendperiod]{scrartcl}

\usepackage{amsmath,amssymb}
\usepackage{iftex}
\ifPDFTeX
  \usepackage[T1]{fontenc}
  \usepackage[utf8]{inputenc}
  \usepackage{textcomp} % provide euro and other symbols
\else % if luatex or xetex
  \usepackage{unicode-math}
  \defaultfontfeatures{Scale=MatchLowercase}
  \defaultfontfeatures[\rmfamily]{Ligatures=TeX,Scale=1}
\fi
\usepackage{lmodern}
\ifPDFTeX\else  
    % xetex/luatex font selection
\fi
% Use upquote if available, for straight quotes in verbatim environments
\IfFileExists{upquote.sty}{\usepackage{upquote}}{}
\IfFileExists{microtype.sty}{% use microtype if available
  \usepackage[]{microtype}
  \UseMicrotypeSet[protrusion]{basicmath} % disable protrusion for tt fonts
}{}
\makeatletter
\@ifundefined{KOMAClassName}{% if non-KOMA class
  \IfFileExists{parskip.sty}{%
    \usepackage{parskip}
  }{% else
    \setlength{\parindent}{0pt}
    \setlength{\parskip}{6pt plus 2pt minus 1pt}}
}{% if KOMA class
  \KOMAoptions{parskip=half}}
\makeatother
\usepackage{xcolor}
\setlength{\emergencystretch}{3em} % prevent overfull lines
\setcounter{secnumdepth}{-\maxdimen} % remove section numbering
% Make \paragraph and \subparagraph free-standing
\makeatletter
\ifx\paragraph\undefined\else
  \let\oldparagraph\paragraph
  \renewcommand{\paragraph}{
    \@ifstar
      \xxxParagraphStar
      \xxxParagraphNoStar
  }
  \newcommand{\xxxParagraphStar}[1]{\oldparagraph*{#1}\mbox{}}
  \newcommand{\xxxParagraphNoStar}[1]{\oldparagraph{#1}\mbox{}}
\fi
\ifx\subparagraph\undefined\else
  \let\oldsubparagraph\subparagraph
  \renewcommand{\subparagraph}{
    \@ifstar
      \xxxSubParagraphStar
      \xxxSubParagraphNoStar
  }
  \newcommand{\xxxSubParagraphStar}[1]{\oldsubparagraph*{#1}\mbox{}}
  \newcommand{\xxxSubParagraphNoStar}[1]{\oldsubparagraph{#1}\mbox{}}
\fi
\makeatother

\usepackage{color}
\usepackage{fancyvrb}
\newcommand{\VerbBar}{|}
\newcommand{\VERB}{\Verb[commandchars=\\\{\}]}
\DefineVerbatimEnvironment{Highlighting}{Verbatim}{commandchars=\\\{\}}
% Add ',fontsize=\small' for more characters per line
\usepackage{framed}
\definecolor{shadecolor}{RGB}{241,243,245}
\newenvironment{Shaded}{\begin{snugshade}}{\end{snugshade}}
\newcommand{\AlertTok}[1]{\textcolor[rgb]{0.68,0.00,0.00}{#1}}
\newcommand{\AnnotationTok}[1]{\textcolor[rgb]{0.37,0.37,0.37}{#1}}
\newcommand{\AttributeTok}[1]{\textcolor[rgb]{0.40,0.45,0.13}{#1}}
\newcommand{\BaseNTok}[1]{\textcolor[rgb]{0.68,0.00,0.00}{#1}}
\newcommand{\BuiltInTok}[1]{\textcolor[rgb]{0.00,0.23,0.31}{#1}}
\newcommand{\CharTok}[1]{\textcolor[rgb]{0.13,0.47,0.30}{#1}}
\newcommand{\CommentTok}[1]{\textcolor[rgb]{0.37,0.37,0.37}{#1}}
\newcommand{\CommentVarTok}[1]{\textcolor[rgb]{0.37,0.37,0.37}{\textit{#1}}}
\newcommand{\ConstantTok}[1]{\textcolor[rgb]{0.56,0.35,0.01}{#1}}
\newcommand{\ControlFlowTok}[1]{\textcolor[rgb]{0.00,0.23,0.31}{\textbf{#1}}}
\newcommand{\DataTypeTok}[1]{\textcolor[rgb]{0.68,0.00,0.00}{#1}}
\newcommand{\DecValTok}[1]{\textcolor[rgb]{0.68,0.00,0.00}{#1}}
\newcommand{\DocumentationTok}[1]{\textcolor[rgb]{0.37,0.37,0.37}{\textit{#1}}}
\newcommand{\ErrorTok}[1]{\textcolor[rgb]{0.68,0.00,0.00}{#1}}
\newcommand{\ExtensionTok}[1]{\textcolor[rgb]{0.00,0.23,0.31}{#1}}
\newcommand{\FloatTok}[1]{\textcolor[rgb]{0.68,0.00,0.00}{#1}}
\newcommand{\FunctionTok}[1]{\textcolor[rgb]{0.28,0.35,0.67}{#1}}
\newcommand{\ImportTok}[1]{\textcolor[rgb]{0.00,0.46,0.62}{#1}}
\newcommand{\InformationTok}[1]{\textcolor[rgb]{0.37,0.37,0.37}{#1}}
\newcommand{\KeywordTok}[1]{\textcolor[rgb]{0.00,0.23,0.31}{\textbf{#1}}}
\newcommand{\NormalTok}[1]{\textcolor[rgb]{0.00,0.23,0.31}{#1}}
\newcommand{\OperatorTok}[1]{\textcolor[rgb]{0.37,0.37,0.37}{#1}}
\newcommand{\OtherTok}[1]{\textcolor[rgb]{0.00,0.23,0.31}{#1}}
\newcommand{\PreprocessorTok}[1]{\textcolor[rgb]{0.68,0.00,0.00}{#1}}
\newcommand{\RegionMarkerTok}[1]{\textcolor[rgb]{0.00,0.23,0.31}{#1}}
\newcommand{\SpecialCharTok}[1]{\textcolor[rgb]{0.37,0.37,0.37}{#1}}
\newcommand{\SpecialStringTok}[1]{\textcolor[rgb]{0.13,0.47,0.30}{#1}}
\newcommand{\StringTok}[1]{\textcolor[rgb]{0.13,0.47,0.30}{#1}}
\newcommand{\VariableTok}[1]{\textcolor[rgb]{0.07,0.07,0.07}{#1}}
\newcommand{\VerbatimStringTok}[1]{\textcolor[rgb]{0.13,0.47,0.30}{#1}}
\newcommand{\WarningTok}[1]{\textcolor[rgb]{0.37,0.37,0.37}{\textit{#1}}}

\providecommand{\tightlist}{%
  \setlength{\itemsep}{0pt}\setlength{\parskip}{0pt}}\usepackage{longtable,booktabs,array}
\usepackage{calc} % for calculating minipage widths
% Correct order of tables after \paragraph or \subparagraph
\usepackage{etoolbox}
\makeatletter
\patchcmd\longtable{\par}{\if@noskipsec\mbox{}\fi\par}{}{}
\makeatother
% Allow footnotes in longtable head/foot
\IfFileExists{footnotehyper.sty}{\usepackage{footnotehyper}}{\usepackage{footnote}}
\makesavenoteenv{longtable}
\usepackage{graphicx}
\makeatletter
\newsavebox\pandoc@box
\newcommand*\pandocbounded[1]{% scales image to fit in text height/width
  \sbox\pandoc@box{#1}%
  \Gscale@div\@tempa{\textheight}{\dimexpr\ht\pandoc@box+\dp\pandoc@box\relax}%
  \Gscale@div\@tempb{\linewidth}{\wd\pandoc@box}%
  \ifdim\@tempb\p@<\@tempa\p@\let\@tempa\@tempb\fi% select the smaller of both
  \ifdim\@tempa\p@<\p@\scalebox{\@tempa}{\usebox\pandoc@box}%
  \else\usebox{\pandoc@box}%
  \fi%
}
% Set default figure placement to htbp
\def\fps@figure{htbp}
\makeatother

\usepackage{float}
\usepackage{tabularray}
\usepackage[normalem]{ulem}
\usepackage{graphicx}
\UseTblrLibrary{booktabs}
\UseTblrLibrary{rotating}
\UseTblrLibrary{siunitx}
\NewTableCommand{\tinytableDefineColor}[3]{\definecolor{#1}{#2}{#3}}
\newcommand{\tinytableTabularrayUnderline}[1]{\underline{#1}}
\newcommand{\tinytableTabularrayStrikeout}[1]{\sout{#1}}
\usepackage{fvextra}
\DefineVerbatimEnvironment{Highlighting}{Verbatim}{breaklines,commandchars=\\\{\}}
\DefineVerbatimEnvironment{OutputCode}{Verbatim}{breaklines,commandchars=\\\{\}}
\KOMAoption{captions}{tableheading}
\makeatletter
\@ifpackageloaded{caption}{}{\usepackage{caption}}
\AtBeginDocument{%
\ifdefined\contentsname
  \renewcommand*\contentsname{Table of contents}
\else
  \newcommand\contentsname{Table of contents}
\fi
\ifdefined\listfigurename
  \renewcommand*\listfigurename{List of Figures}
\else
  \newcommand\listfigurename{List of Figures}
\fi
\ifdefined\listtablename
  \renewcommand*\listtablename{List of Tables}
\else
  \newcommand\listtablename{List of Tables}
\fi
\ifdefined\figurename
  \renewcommand*\figurename{Figure}
\else
  \newcommand\figurename{Figure}
\fi
\ifdefined\tablename
  \renewcommand*\tablename{Table}
\else
  \newcommand\tablename{Table}
\fi
}
\@ifpackageloaded{float}{}{\usepackage{float}}
\floatstyle{ruled}
\@ifundefined{c@chapter}{\newfloat{codelisting}{h}{lop}}{\newfloat{codelisting}{h}{lop}[chapter]}
\floatname{codelisting}{Listing}
\newcommand*\listoflistings{\listof{codelisting}{List of Listings}}
\makeatother
\makeatletter
\makeatother
\makeatletter
\@ifpackageloaded{caption}{}{\usepackage{caption}}
\@ifpackageloaded{subcaption}{}{\usepackage{subcaption}}
\makeatother

\usepackage{bookmark}

\IfFileExists{xurl.sty}{\usepackage{xurl}}{} % add URL line breaks if available
\urlstyle{same} % disable monospaced font for URLs
\hypersetup{
  pdftitle={Problem Set 2},
  pdfauthor={Hyoungchul Kim},
  colorlinks=true,
  linkcolor={blue},
  filecolor={Maroon},
  citecolor={Blue},
  urlcolor={Blue},
  pdfcreator={LaTeX via pandoc}}


\title{Problem Set 2}
\author{Hyoungchul Kim}
\date{2025-03-02}

\begin{document}
\maketitle


First read in some necessary programming packages for analysis:

\begin{Shaded}
\begin{Highlighting}[]
\CommentTok{\# Load libraries}
\ControlFlowTok{if}\NormalTok{ (}\SpecialCharTok{!}\FunctionTok{require}\NormalTok{(pacman)) }\FunctionTok{install.packages}\NormalTok{(}\StringTok{"pacman"}\NormalTok{)}
\FunctionTok{p\_load}\NormalTok{(tidyverse, data.table, fixest, modelsummary, texreg)}
\end{Highlighting}
\end{Shaded}

\section{Theory questions}\label{theory-questions}

\subsection{a.}\label{a.}

We learned in class that 2SLS coefficients in finite sample is biased.
To be exact, Hahn and Hausman (\emph{Ecma}, 2002) showed that
\(\mathbb{E}(b_{2SLS} - \beta) \approx \frac{\sigma_{\eta \xi}}{\sigma^2_\xi} \frac{1}{F+1}\)
where \(F\) is the first stagef-statistics, a proxy for the predictive
strength of the instrument effect. since \(F\) is in the denominator, we
can see that higher predictive power will decrease the bias. On the
other hand, the predictive strength of the instrument does not affect
the consistency of the IV estimator. As long as we have the exclusion
condition satisfied, the probability limit of the IV estimator converges
to the estimand.

\subsection{b.}\label{b.}

Since LATE is still a type of IV, we need the exclusion and relativity
condition to hold. That is, the IV should be correlated with the
treatment varible of interest and should only affect the outcome through
the treatment variable. (Note that SUTVA, random assignment is also
necessary for the IV method to hold).

One extra assumption we need is the \textbf{monotonicity condition}.
This just means that there are no defier.

Now let's prove the LATE theorem under these maintained assumptions:

Using assumptions above, we can get:

\begin{align}
  \beta_{late} &= \frac{\mathbb{E}(Y_i \mid Z_i = 1) - \mathbb{E}(Y_i \mid Z_i = 0)}{\mathbb{E}(D_i \mid Z_i = 1) - \mathbb{E}(D_i \mid Z_i = 0)}\\
   &= \frac{\mathbb{E}[Y_{1i} - Y_{0i} \mid \text{compliers}]P(\text{compliers})}{P(\text{compliers})}\\
   &= \mathbb{E}[Y_{1i} - Y_{0i} \mid \text{compliers}]
\end{align}

So this shows that by using the assumption we can get the ATE for the
compliers in our data.

\subsection{c.}\label{c.}

You can easily check from the question b that the monotonicity is
crucial for retrieving LATE. The nominator in the question b is the
reduced form estimate. Without monotonicity, we cannot have the reduced
form estimate to equal the change in potential outcome effect for
compliers. You can also see that the denominator implies the usefulness
of monotonicity in first stage. If we assume there is defiers, the first
stage estimate cannot be the probability of being the complier. Thus
monotonicity is crucial to pin down the LATE estimate.

\subsection{d.}\label{d.}

If there is no always takers, this means all of our data will be made up
of compliers (under the assumption of monotonicity). So this implies
that \(\mathbb{E} [D_i \mid Z_i = 0] = 0\). Then using the fact that
\(\mathbb{E}[Y_i \mid Z_i = 1] = \mathbb{E}[Y_{0i}]\) and
\(\mathbb{E}[Y_i \mid Z_i = 1] = \mathbb{E}[Y_{0i} + (Y_{1i} - Y_{0i}) D_i \mid Z_i = 1]\)
from the lecture, we get:

\begin{align}
 \beta_{late} &= \frac{\mathbb{E} [(Y_{1i}-Y_{0i}) D_i \mid Z_i = 1]}{P[D_i = 1 \mid Z_i =1]}\\
  &= \frac{\mathbb{E}[(Y_{1i} - Y_{0i}) (1) \mid D_i =1, Z_i =1] P[D_i = 1 \mid Z_i = 1]+ 0}{P[D_i = 1 \mid Z_i = 1]}\\
  &= \mathbb{E}[(Y_{1i} - Y_{0i} \mid D_i =1, Z_i = 1)]\\
  &= \mathbb{E}[(Y_{1i} - Y_{0i}) \mid D_i = 1] = ATT
\end{align}

\subsection{e.}\label{e.}

Unlike LATE, MTC allows sorting on unobserved treatment effects. For
LATE, if there is unobserved soring, it might not recover the ATE for
compliers. MTA in some way alleviates this issue by allowing treatment
probability to be correlated with treatment effects. It is also helpful
if you want to recover population treatment effect parameters like ATT
and not jus the LATE. But we need to assume some functional form
restrictions. For example, we assume treatment effect is additive and
separable. Also, we assume outcome Y depends on X in a linear, additive
form. Using these assumptions, we can write MTC as sum of observed and
unobserved components:
\(MTE(X=x, U_D = p) = x'(\beta_1 - \beta_0) + \mathbb{E}(u_{1i} - u_{0i} \mid p)\).
Then we can express
\(\mathbb{E}[Y_i \mid X = x, P = p] = x' \beta_0 + x' (\beta_1 - \beta_0)p + K(p)\),
where laster term on RHS is a polynomial that approximates the
unobserved portion of Y. Then we can estimate the MTE by taking the
derivative with respect to p.

\section{Numerical questions}\label{numerical-questions}

\subsection{a.}\label{a.-1}

First, we read in the data.

\begin{Shaded}
\begin{Highlighting}[]
\NormalTok{data\_des }\OtherTok{\textless{}{-}}\NormalTok{ haven}\SpecialCharTok{::}\FunctionTok{read\_dta}\NormalTok{(}\StringTok{"../data/OHIE\_Public\_Use\_Files/OHIE\_Data/oregonhie\_descriptive\_vars.dta"}\NormalTok{)}
\NormalTok{data\_state }\OtherTok{\textless{}{-}}\NormalTok{ haven}\SpecialCharTok{::}\FunctionTok{read\_dta}\NormalTok{(}\StringTok{"../data/OHIE\_Public\_Use\_Files/OHIE\_Data/oregonhie\_stateprograms\_vars.dta"}\NormalTok{)}
\NormalTok{data\_0 }\OtherTok{\textless{}{-}}\NormalTok{ haven}\SpecialCharTok{::}\FunctionTok{read\_dta}\NormalTok{(}\StringTok{"../data/OHIE\_Public\_Use\_Files/OHIE\_Data/oregonhie\_survey0m\_vars.dta"}\NormalTok{)}
\NormalTok{data\_12 }\OtherTok{\textless{}{-}}\NormalTok{ haven}\SpecialCharTok{::}\FunctionTok{read\_dta}\NormalTok{(}\StringTok{"../data/OHIE\_Public\_Use\_Files/OHIE\_Data/oregonhie\_survey12m\_vars.dta"}\NormalTok{)}

\NormalTok{data }\OtherTok{\textless{}{-}}\NormalTok{ data\_des }\SpecialCharTok{\%\textgreater{}\%} 
  \FunctionTok{left\_join}\NormalTok{(data\_state, }\AttributeTok{by=}\StringTok{"person\_id"}\NormalTok{) }\SpecialCharTok{\%\textgreater{}\%} 
  \FunctionTok{left\_join}\NormalTok{(data\_0, }\AttributeTok{by =} \StringTok{"person\_id"}\NormalTok{) }\SpecialCharTok{\%\textgreater{}\%} 
  \FunctionTok{left\_join}\NormalTok{(data\_12, }\AttributeTok{by =} \StringTok{"person\_id"}\NormalTok{)}
\end{Highlighting}
\end{Shaded}

Now we get the balance table by treatment status. We do it for full
sample:

\subsubsection{Full sample}\label{full-sample}

\begin{Shaded}
\begin{Highlighting}[]
\CommentTok{\# Convert treatment to a factor and then assign new labels}
\NormalTok{data }\OtherTok{\textless{}{-}} \FunctionTok{as.data.table}\NormalTok{(data)}
\NormalTok{data[, age }\SpecialCharTok{:}\ErrorTok{=} \DecValTok{2009} \SpecialCharTok{{-}}\NormalTok{ birthyear\_list]}
\NormalTok{data[, treatment }\SpecialCharTok{:}\ErrorTok{=} \FunctionTok{as\_factor}\NormalTok{(treatment)]}
\NormalTok{data[, treatment }\SpecialCharTok{:}\ErrorTok{=} \FunctionTok{fcase}\NormalTok{(}
\NormalTok{  treatment }\SpecialCharTok{==} \StringTok{"Selected"}\NormalTok{, }\StringTok{"treat"}\NormalTok{,}
\NormalTok{  treatment }\SpecialCharTok{==} \StringTok{"Not selected"}\NormalTok{, }\StringTok{"control"}
\NormalTok{)]}

\NormalTok{mean\_na }\OtherTok{\textless{}{-}} \ControlFlowTok{function}\NormalTok{(x) }\FunctionTok{mean}\NormalTok{(x, }\AttributeTok{na.rm=}\NormalTok{T)}

\FunctionTok{datasummary}\NormalTok{(age }\SpecialCharTok{+}\NormalTok{ female\_list }\SpecialCharTok{+}\NormalTok{ english\_list }\SpecialCharTok{+}\NormalTok{ zip\_msa\_list }\SpecialCharTok{\textasciitilde{}}\NormalTok{ treatment }\SpecialCharTok{*}\NormalTok{ mean\_na, }
  \AttributeTok{data =}\NormalTok{ data,}
  \AttributeTok{title =} \StringTok{"Balance table (full sample)"}\NormalTok{,}
  \AttributeTok{fmt =} \DecValTok{3}\NormalTok{)}
\end{Highlighting}
\end{Shaded}

\begin{table}
\centering
\begin{talltblr}[         %% tabularray outer open
caption={Balance table (full sample)},
]                     %% tabularray outer close
{                     %% tabularray inner open
colspec={Q[]Q[]Q[]},
column{1}={}{halign=l,},
column{2,3}={}{halign=r,},
}                     %% tabularray inner close
\toprule
& control & treat \\ \midrule %% TinyTableHeader
age                                                                & \num{41.002} & \num{40.804} \\
Female: lottery list data                                          & \num{0.557}  & \num{0.541}  \\
Individual requested english-language materials: lottery list data & \num{0.922}  & \num{0.902}  \\
Zip code from lottery list is a metropolitan statistical area      & \num{0.773}  & \num{0.764}  \\
\bottomrule
\end{talltblr}
\end{table}

Next we do it for people who responded to the survey:

\subsubsection{Survery responder}\label{survery-responder}

\begin{Shaded}
\begin{Highlighting}[]
\FunctionTok{datasummary}\NormalTok{(age }\SpecialCharTok{+}\NormalTok{ female\_list }\SpecialCharTok{+}\NormalTok{ english\_list }\SpecialCharTok{+}\NormalTok{ zip\_msa\_list }\SpecialCharTok{+}\NormalTok{ employ\_hrs\_12m }\SpecialCharTok{+}\NormalTok{ edu\_12m }\SpecialCharTok{\textasciitilde{}}\NormalTok{ treatment }\SpecialCharTok{*}\NormalTok{ mean\_na,}
  \AttributeTok{data =}\NormalTok{ data,}
  \AttributeTok{title =} \StringTok{"Balance table (survey responders only)"}\NormalTok{,}
  \AttributeTok{fmt =} \DecValTok{3}\NormalTok{)}
\end{Highlighting}
\end{Shaded}

\begin{table}
\centering
\begin{talltblr}[         %% tabularray outer open
caption={Balance table (survey responders only)},
]                     %% tabularray outer close
{                     %% tabularray inner open
colspec={Q[]Q[]Q[]},
column{1}={}{halign=l,},
column{2,3}={}{halign=r,},
}                     %% tabularray inner close
\toprule
& control & treat \\ \midrule %% TinyTableHeader
age                                                                & \num{41.002} & \num{40.804} \\
Female: lottery list data                                          & \num{0.557}  & \num{0.541}  \\
Individual requested english-language materials: lottery list data & \num{0.922}  & \num{0.902}  \\
Zip code from lottery list is a metropolitan statistical area      & \num{0.773}  & \num{0.764}  \\
Average hrs worked/week                                            & \num{2.059}  & \num{2.106}  \\
Highest level of education completed                               & \num{2.280}  & \num{2.274}  \\
\bottomrule
\end{talltblr}
\end{table}

We can see that there is not much significant differences between
control group and treatment group. This seems to imply that
randomization is quite well done.

Now let's check ITT:

\begin{Shaded}
\begin{Highlighting}[]
\NormalTok{itt1 }\OtherTok{\textless{}{-}} \FunctionTok{feols}\NormalTok{(hosp\_any\_12m }\SpecialCharTok{\textasciitilde{}}\NormalTok{ treatment }\SpecialCharTok{+}\NormalTok{ hhsize\_12m }\SpecialCharTok{+}\NormalTok{ draw\_lottery }\SpecialCharTok{+}\NormalTok{ hosp\_any\_0m }\SpecialCharTok{+}\NormalTok{ er\_any\_0m, }\AttributeTok{weights =} \SpecialCharTok{\textasciitilde{}}\NormalTok{ weight\_12m, }\AttributeTok{data =}\NormalTok{ data, }\AttributeTok{cluster =} \SpecialCharTok{\textasciitilde{}}\NormalTok{ household\_id)}

\NormalTok{itt2 }\OtherTok{\textless{}{-}} \FunctionTok{feols}\NormalTok{(er\_any\_12m }\SpecialCharTok{\textasciitilde{}}\NormalTok{ treatment }\SpecialCharTok{+}\NormalTok{ hhsize\_12m }\SpecialCharTok{+}\NormalTok{ draw\_lottery }\SpecialCharTok{+}\NormalTok{hosp\_any\_0m }\SpecialCharTok{+}\NormalTok{ er\_any\_0m, }\AttributeTok{weights =} \SpecialCharTok{\textasciitilde{}}\NormalTok{ weight\_12m, }\AttributeTok{data =}\NormalTok{ data, }\AttributeTok{cluster =} \SpecialCharTok{\textasciitilde{}}\NormalTok{ household\_id)}

\FunctionTok{etable}\NormalTok{(itt1, itt2)}
\end{Highlighting}
\end{Shaded}

\begin{verbatim}
                              itt1               itt2
Dependent Var.:       hosp_any_12m         er_any_12m
                                                     
Constant        0.0502*** (0.0059) 0.1692*** (0.0096)
treatmenttreat    -0.0001 (0.0044)   -0.0002 (0.0074)
hhsize_12m        -0.0003 (0.0003)  -0.0002. (0.0001)
draw_lottery      -0.0006 (0.0010)   -0.0011 (0.0016)
hosp_any_0m     0.1513*** (0.0157)   0.0387* (0.0188)
er_any_0m       0.0446*** (0.0061) 0.2954*** (0.0105)
_______________ __________________ __________________
S.E.: Clustered   by: household_id   by: household_id
Observations                15,454             15,437
R2                         0.03480            0.09515
Adj. R2                    0.03449            0.09486
---
Signif. codes: 0 '***' 0.001 '**' 0.01 '*' 0.05 '.' 0.1 ' ' 1
\end{verbatim}

You can see that the estimate is positive. This makes sense as winning
the lottery makes you more likely to get a health care insurance. In a
way, this could be understood as policy relevant as this shows you the
compliance rate of people the government wants to treat. This estimate
implies the magnitude of how likely people will respond to the treatment
(lottery) policy.

\subsection{b.}\label{b.-1}

Using OLS is problematic as people receive medicaid and people who don't
are different set of people. That is, the potential outcome (health)
will be correlated with the treatment of having medicaid. Thus running
OLS is give us a biased result.

Let's look at a balance table between those who received medicaid and
the remaining people.

\clearpage

While there is too big difference, you can see some characteristics
differ between two groups than the previous randomization (gender ratio,
average hours worked, etc).

\subsection{c.}\label{c.-1}

I will use first use the ``ever on medicaid'' as the measures of
coverage.

Now I will use

\subsection{d.}\label{d.-1}

For monotonicity, I guess it would be assuming that people who got into
lottery will not have tendency to not get a health insurance. I think
monotonicity would hold in this case as the lottery is just giving you
chance to apply for health care (OHS Standard). Thus it will at least
weakly make people apply for health care and get one.

Also, exclusion in this case would mean the lottery will only affect the
outcome (health and hospital related outcome) only through the treatment
(getting a health care). This also seems reasonable as lottery is just
like an experiment. Thus it is very unlikely it will directly affect the
outcome.

\subsection{e.}\label{e.-1}

\subsection{f.}\label{f.}

\subsection{g.}\label{g.}




\end{document}
