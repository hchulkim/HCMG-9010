% Options for packages loaded elsewhere
\PassOptionsToPackage{unicode}{hyperref}
\PassOptionsToPackage{hyphens}{url}
\PassOptionsToPackage{dvipsnames,svgnames,x11names}{xcolor}
%
\documentclass[
  letterpaper,
  DIV=11,
  numbers=noendperiod]{scrartcl}

\usepackage{amsmath,amssymb}
\usepackage{iftex}
\ifPDFTeX
  \usepackage[T1]{fontenc}
  \usepackage[utf8]{inputenc}
  \usepackage{textcomp} % provide euro and other symbols
\else % if luatex or xetex
  \usepackage{unicode-math}
  \defaultfontfeatures{Scale=MatchLowercase}
  \defaultfontfeatures[\rmfamily]{Ligatures=TeX,Scale=1}
\fi
\usepackage{lmodern}
\ifPDFTeX\else  
    % xetex/luatex font selection
\fi
% Use upquote if available, for straight quotes in verbatim environments
\IfFileExists{upquote.sty}{\usepackage{upquote}}{}
\IfFileExists{microtype.sty}{% use microtype if available
  \usepackage[]{microtype}
  \UseMicrotypeSet[protrusion]{basicmath} % disable protrusion for tt fonts
}{}
\makeatletter
\@ifundefined{KOMAClassName}{% if non-KOMA class
  \IfFileExists{parskip.sty}{%
    \usepackage{parskip}
  }{% else
    \setlength{\parindent}{0pt}
    \setlength{\parskip}{6pt plus 2pt minus 1pt}}
}{% if KOMA class
  \KOMAoptions{parskip=half}}
\makeatother
\usepackage{xcolor}
\setlength{\emergencystretch}{3em} % prevent overfull lines
\setcounter{secnumdepth}{-\maxdimen} % remove section numbering
% Make \paragraph and \subparagraph free-standing
\makeatletter
\ifx\paragraph\undefined\else
  \let\oldparagraph\paragraph
  \renewcommand{\paragraph}{
    \@ifstar
      \xxxParagraphStar
      \xxxParagraphNoStar
  }
  \newcommand{\xxxParagraphStar}[1]{\oldparagraph*{#1}\mbox{}}
  \newcommand{\xxxParagraphNoStar}[1]{\oldparagraph{#1}\mbox{}}
\fi
\ifx\subparagraph\undefined\else
  \let\oldsubparagraph\subparagraph
  \renewcommand{\subparagraph}{
    \@ifstar
      \xxxSubParagraphStar
      \xxxSubParagraphNoStar
  }
  \newcommand{\xxxSubParagraphStar}[1]{\oldsubparagraph*{#1}\mbox{}}
  \newcommand{\xxxSubParagraphNoStar}[1]{\oldsubparagraph{#1}\mbox{}}
\fi
\makeatother


\providecommand{\tightlist}{%
  \setlength{\itemsep}{0pt}\setlength{\parskip}{0pt}}\usepackage{longtable,booktabs,array}
\usepackage{calc} % for calculating minipage widths
% Correct order of tables after \paragraph or \subparagraph
\usepackage{etoolbox}
\makeatletter
\patchcmd\longtable{\par}{\if@noskipsec\mbox{}\fi\par}{}{}
\makeatother
% Allow footnotes in longtable head/foot
\IfFileExists{footnotehyper.sty}{\usepackage{footnotehyper}}{\usepackage{footnote}}
\makesavenoteenv{longtable}
\usepackage{graphicx}
\makeatletter
\newsavebox\pandoc@box
\newcommand*\pandocbounded[1]{% scales image to fit in text height/width
  \sbox\pandoc@box{#1}%
  \Gscale@div\@tempa{\textheight}{\dimexpr\ht\pandoc@box+\dp\pandoc@box\relax}%
  \Gscale@div\@tempb{\linewidth}{\wd\pandoc@box}%
  \ifdim\@tempb\p@<\@tempa\p@\let\@tempa\@tempb\fi% select the smaller of both
  \ifdim\@tempa\p@<\p@\scalebox{\@tempa}{\usebox\pandoc@box}%
  \else\usebox{\pandoc@box}%
  \fi%
}
% Set default figure placement to htbp
\def\fps@figure{htbp}
\makeatother

\usepackage{fvextra}
\DefineVerbatimEnvironment{Highlighting}{Verbatim}{breaklines,commandchars=\\\{\}}
\DefineVerbatimEnvironment{OutputCode}{Verbatim}{breaklines,commandchars=\\\{\}}
\KOMAoption{captions}{tableheading}
\makeatletter
\@ifpackageloaded{caption}{}{\usepackage{caption}}
\AtBeginDocument{%
\ifdefined\contentsname
  \renewcommand*\contentsname{Table of contents}
\else
  \newcommand\contentsname{Table of contents}
\fi
\ifdefined\listfigurename
  \renewcommand*\listfigurename{List of Figures}
\else
  \newcommand\listfigurename{List of Figures}
\fi
\ifdefined\listtablename
  \renewcommand*\listtablename{List of Tables}
\else
  \newcommand\listtablename{List of Tables}
\fi
\ifdefined\figurename
  \renewcommand*\figurename{Figure}
\else
  \newcommand\figurename{Figure}
\fi
\ifdefined\tablename
  \renewcommand*\tablename{Table}
\else
  \newcommand\tablename{Table}
\fi
}
\@ifpackageloaded{float}{}{\usepackage{float}}
\floatstyle{ruled}
\@ifundefined{c@chapter}{\newfloat{codelisting}{h}{lop}}{\newfloat{codelisting}{h}{lop}[chapter]}
\floatname{codelisting}{Listing}
\newcommand*\listoflistings{\listof{codelisting}{List of Listings}}
\makeatother
\makeatletter
\makeatother
\makeatletter
\@ifpackageloaded{caption}{}{\usepackage{caption}}
\@ifpackageloaded{subcaption}{}{\usepackage{subcaption}}
\makeatother

\usepackage{bookmark}

\IfFileExists{xurl.sty}{\usepackage{xurl}}{} % add URL line breaks if available
\urlstyle{same} % disable monospaced font for URLs
\hypersetup{
  pdftitle={Proposal},
  pdfauthor={Hyoungchul Kim},
  colorlinks=true,
  linkcolor={blue},
  filecolor={Maroon},
  citecolor={Blue},
  urlcolor={Blue},
  pdfcreator={LaTeX via pandoc}}


\title{Proposal}
\author{Hyoungchul Kim}
\date{2025-04-24}

\begin{document}
\maketitle


\subsection{1. Introduction and
Motivation}\label{introduction-and-motivation}

In a world characterized by unprecedented human mobility and political
uncertainty, understanding how immigration interacts with the incidence
of armed conflict is both timely and policy-relevant. While some fear
that rising immigration may import external tensions or provoke native
resistance, others argue that increased diversity might actually deter
conflict by raising the economic and social opportunity costs of
violence.

This research seeks to unpack the nuanced relationship between
immigration and armed conflicts. Does immigration exacerbate domestic
unrest or mitigate violence? Do these effects vary across types of
conflict---such as interstate versus intrastate? Moreover, is there an
indirect mechanism at play, whereby a country's immigrant demographics
influence the strategic calculus of other nations?

This project aims to shed light on these questions using a novel global
dataset and a well-established econometric strategy. Given the current
debates surrounding migration policy and national security, the findings
have the potential to influence both academic discourse and real-world
decision-making.

\subsection{2. Research Questions and Conceptual
Framework}\label{research-questions-and-conceptual-framework}

The core research question is:

\begin{quote}
\textbf{Does immigration increase or decrease the likelihood and
intensity of armed conflict?}
\end{quote}

To address this, the project explores several layers:

\begin{itemize}
\item
  \textbf{Heterogeneity by conflict type:} Does immigration influence
  intrastate conflicts (e.g., civil wars) differently than interstate
  ones (e.g., international wars)?
\item
  \textbf{Direct vs.~indirect channels:} Does the presence of immigrants
  affect the host country's internal dynamics, or does it also impact
  the strategic behavior of other countries?
\end{itemize}

From a theoretical standpoint, the direct mechanisms might involve
either increased ethnic tensions (leading to conflict) or economic
integration (reducing conflict). Indirectly, countries with large
immigrant populations from a potential adversary might be less inclined
to initiate hostilities, fearing domestic backlash or divided loyalties.

\subsection{3. Empirical Strategy and
Data}\label{empirical-strategy-and-data}

\subsubsection{Data Sources}\label{data-sources}

\begin{itemize}
\item
  \textbf{Immigration:}

  \begin{itemize}
  \item
    UN Global Migration Database (1990--2024, 5-year intervals) -- used
    for main regressors.
  \item
    World Bank Migration Data (1960--2000, decadal) -- used to construct
    historical shares for the IV.
  \end{itemize}
\item
  \textbf{Armed Conflicts:}

  \begin{itemize}
  \item
    UCDP/PRIO Dataset (1946--2022) -- provides annual data on conflict
    incidence and type, using a 25-battle-death threshold.
  \item
    Controls: CEPII (trade, GDP, WTO/GATT membership, etc.).
  \end{itemize}
\end{itemize}

The analysis uses country-level panels aggregated at 5-year intervals,
focusing on the 1990--2022 period.

\subsubsection{Econometric Design}\label{econometric-design}

The paper uses a \textbf{shift-share instrumental variable (IV)}
strategy. This approach isolates exogenous variation in immigration by
exploiting historical settlement patterns and global shocks.

Formally, the IV for immigration share in country \emph{i} at time
\emph{t} is:

\[
IMM_{it} = IMM_{i,t-1} + \sum_j (World{j,t-1} × (Country_{ij,1960} / Country_{i,1960}))
\]

Where:

\begin{itemize}
\item
  \textbf{World} denotes global outflows from country \emph{j} at time
  \emph{t}.
\item
  \textbf{Share} is the fixed distribution of immigrants from \emph{j}
  to \emph{i} in 1960.
\end{itemize}

The identifying assumption is that immigration shifts are exogenous once
historical shares are fixed. The strategy mimics a natural experiment
driven by global push factors.

\subsection{4. Preliminary Results and
Challenges}\label{preliminary-results-and-challenges}

Initial 2SLS regressions suggest that a higher share of immigrants
correlates with a \textbf{reduction in intrastate conflict}, though
robustness remains a concern:

\begin{itemize}
\item
  Coefficient on immigration share is negative and significant in
  baseline specifications.
\item
  First-stage F-statistics are high, indicating strong instrument
  performance, although some anomalies in magnitude suggest possible
  measurement issues.
\end{itemize}

Some regressions yield unstable coefficients under alternative
specifications, prompting exploration of:

\begin{itemize}
\item
  \textbf{Alternative IV constructions} (e.g., modifying ``shift'' and
  ``share'' definitions).
\item
  \textbf{Supplementary conflict datasets} like Correlates of War.
\item
  \textbf{Improved harmonization of immigration data}, since current
  analysis mixes UN and World Bank sources.
\end{itemize}

\subsection{5. Limitations and Next
Steps}\label{limitations-and-next-steps}

This is a challenging empirical question. Conflict data is noisy and
multifaceted, and migration flows are difficult to predict with
precision. Key issues to address in the next stage include:

\begin{itemize}
\item
  Better handling of data integration across sources.
\item
  Testing additional robustness checks (placebo outcomes, falsification
  tests).
\item
  Exploring micro-level dynamics (e.g., city-level conflicts or ethnic
  tensions).
\end{itemize}

If the relationship between immigration and conflict is not strong, that
in itself is a valuable finding---it challenges alarmist narratives and
supports the idea that migration may not undermine peace.

\subsection{6. Conclusion}\label{conclusion}

This project offers an ambitious but feasible attempt to quantify the
impact of immigration on armed conflict using a global panel and a
credible identification strategy. While preliminary findings point
toward a pacifying effect of immigration---particularly on intrastate
conflict---more work is needed to address specification issues and data
consistency.

The broader goal is not only to contribute to the academic literature
but to inform a more balanced public discussion on the implications of
immigration in a globalized world.




\end{document}
